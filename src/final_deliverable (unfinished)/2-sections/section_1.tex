
% Section 0

\section*{Abstract}

abstract here...

abstract here...

abstract here...

abstract here...

abstract here...

abstract here...

abstract here...


% First section

\section{Introduction}


Flying taxis are expected to serve as an alternative to ground transportation to alleviate traffic congestion in metropolitan cities. Several transportation pioneers and airline manufacturers are preparing to launch their Urban Air Mobility (\acs{UAM}) services in the near future. Indeed, Airbus Helicopters is currently working on new electric flying taxis as a part of the \textit{CityAirbus Nextgen} project~\cite{airbus2025}. Their flying taxis are expected to operate in 2025. Apart from Airbus, Uber is also working on an air taxi service called \textit{Uber Elevate}, which is estimated to lunch in 2023~\cite{rajendran2019}.

Due to the dynamic nature of the problem, the companies proposing air taxi service will deal with several real-time decisions. Such decisions include
\begin{enumerate*}[label=(\roman*)]
	\item evaluating different candidate trips and scheduling the flying taxis so as to optimize a given objective (\textit{e.g.}, reduce the operational costs, serve the maximum number of demands, etc.),
	\item dynamic estimation of the market demands, and
	
	\item battery charging management as well as other maintenance related-issues~\cite{rajendran2020}.
\end{enumerate*}

In this work, we are interested in the real-time management of a fleet of flying taxis. The management includes the dynamic scheduling of the air taxis and the battery charging handling. This work is one of the few that considers the (more realistic) real-time problem, where decisions have to be made dynamically to accommodate the new changes in the air taxi environment. 

A common strategy used in the literature to tackle the dynamic scheduling is the \textit{Rolling Horizon} (\acs{RH}) approach. The latter usually subdivides the scheduling horizon into smaller sub-horizons, and then it solves the static problem on each sub-horizon. These strategies are used in the literature to tackle (the dynamic version of) two well-know optimization problem that are \emph{similar} to our problem, namely: the Job-Shop Scheduling Problem (\acs{JSSP}) and the Vehicle Routing Problem (\acs{VRP}). 

The analogy between a \acs{JSSP} and the scheduling of air taxis can be described as follows. The machines represent the flying taxis, and the jobs represent the demands. The trip duration of a given demand can be interpreted as the processing time of a job in the \acs{JSSP} framework. A fundamental aspect of the flying taxis framework is the battery charging; the latter can be translated to the \acs{JSSP} model as the machine breakdown, \textit{i.e.,} a specified period of time during which a machine is is out of order.

In addition, the scheduling of air taxis is clearly similar to the \acs{VRP}. In the two frameworks, we have vehicles to dispatch and clients to serve. The battery charging of an air taxi is translated to the VRP framework as the vehicle refueling.

As a consequence of these analogies, we can derive two conclusions. On the one hand, the problem of scheduling flying taxis, the \acs{JSSP}, and the \acs{VRP} have the same complexity: they are NP-hard problems (\acs{JSSP} and the \acs{VRP} are already proven to be NP-hard~\cite{mohan2019,song2016}). On the other hand, we can use the literature of the dynamic \acs{JSSP} and the \acs{VRP} to gain insights on how the real-time problems are handled. The latter is the objective of the next section.


